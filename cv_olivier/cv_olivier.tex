%%%%%%%%%%%%%%%%%%%%%%%%%%%%%%%%%%%%%%%%%
% Medium Length Professional CV
% LaTeX Template
% Version 2.0 (8/5/13)
%
% This template has been downloaded from:
% http://www.LaTeXTemplates.com
%
% Original author:
% Rishi Shah 
%
% Important note:
% This template requires the resume.cls file to be in the same directory as the
% .tex file. The resume.cls file provides the resume style used for structuring the
% document.
%
%%%%%%%%%%%%%%%%%%%%%%%%%%%%%%%%%%%%%%%%%

%----------------------------------------------------------------------------------------
%	PACKAGES AND OTHER DOCUMENT CONFIGURATIONS
%----------------------------------------------------------------------------------------

\documentclass{resume} % Use the custom resume.cls style

\usepackage[left=0.75in,top=0.6in,right=0.75in,bottom=0.6in]{geometry} % Document margins
\newcommand{\tab}[1]{\hspace{.2667\textwidth}\rlap{#1}}
\newcommand{\itab}[1]{\hspace{0em}\rlap{#1}}
\name{Olivier St-Louis} % Your name
\address{(+1) 819 960-0867 \\ olivieri02@hotmail.com} % Your phone number and email

\begin{document}

%----------------------------------------------------------------------------------------
%	SECTION Formations
%----------------------------------------------------------------------------------------

\begin{rSection}{Formations}

{\bf Université Laval}, Québec \hfill {\em Août 2025 - Aujourd'hui}\\
{\bf Maîtrise en science politique}, Directeur : {\em Thierry Giasson}\\
Politique québécoise et canadienne\\\\
{\bf Université Laval}, Québec \hfill {\em Août 2022 - Août 2025}\\
{\bf Baccalauréat intégré en affaires publiques et}\\ 
{\bf relations internationales}\\
Profil international\\\\
{\bf Cégep de Victoriaville}, Victoriaville \hfill {\em Août 2020 - Mai 2022} 
\\ Sciences humaines - Développement des communautés
\\

\end{rSection}

%--------------------------------------------------------------------------------
%    Expériences
%-----------------------------------------------------------------------------------------------
\begin{rSection}{Expériences}
{\bf Étudiant conseiller en rétro-information au} \hfill {\em Septembre 2023 - Aujourd'hui}\\
{\bf ministère du Conseil exécutif}\\
- Parcourir les différentes plateformes de diffusion de nouvelles afin d’identifier l’information pertinente en lien avec l’action gouvernementale;\\
- Suivre l’évolution du traitement journalistique des sujets chauds et qui risquent d’avoir un écho important dans l’actualité, et ce, en tenant compte de la réalité et de la mission de chacun des ministères;\\
- Résumer les différentes nouvelles de l’actualité en identifiant leurs idées maîtresses et leur positionnement dans le fil de l’actualité.\\\\
{\bf Guide-interprète à la Résidence du Gouverneur} \hfill {\em Mai 2023 - Décembre 2023}\\
{\bf général à la Citadelle}\\
- Accueillir des invités et des dignitaires lors d’événements officiels en présence de la gouverneure générale (à l’occasion);\\
- Offrir des visites guidées qui permettent aux visiteurs d’en apprendre davantage sur les responsabilités exercées par le gouverneur général, tout en découvrant l’histoire, l’architecture et les collections de la résidence.\\\\
{\bf Associé dans les départements de la plomberie,} \hfill {\em Mai 2022 - Mai 2023}\\
{\bf de l’électricité et du saisonnier chez Home Depot}\\
- Accueillir les clients et les conseiller dans leur recherche et/ou besoin;\\
- Maintenance du département pour s’assurer que tout soit propre et sécuritaire.\\\\
{\bf Parcours chez McDonald's} \hfill {\em Octobre 2017 - Décembre 2021}\\
- De août 2021 à décembre 2021, assistant-gérant à temps partiel. Faire les horaires de près de 120 employés en répartissant les forces selon les journées et les ventes prévues. Retourner des appels concernant les plaintes en s’assurant de donner un service client hors-pair;\\
- De mars 2020 à août 2021, chef de quart à temps partiel. Organiser le plancher pour qu’il fonctionne optimalement et rôle de coach avec les employés (faire du perfectionnement).\\
- Préparer, assembler et donner les commandes aux clients.\\\\
\end{rSection}

%--------------------------------------------------------------------------------
%    Implications bénévoles
%-----------------------------------------------------------------------------------------------

\begin{rSection}{Implications bénévoles}
{\bf Mentor de délégation aux Jeux de la science politique} \hfill {\em Juin 2025 - Janvier 2026}\\
Délégation de l'Université Laval\\\\
{\bf Membre de l'équipe Québec Forte et Fière} \hfill {\em Juillet 2025 - Novembre 2025}\\
Opération de terrain - Vanier-Duberger\\\\
{\bf Coordonnateur du Comité Sport et plein air du BIAPRI} \hfill {\em Septembre 2023 - Mai 2024}\\\\
{\bf Implications au Cégep de Victoriaville} \hfill {\em Août 2021 - Décembre 2021}\\
- Mentorat auprès des nouveaux étudiants en sciences humaines;\\
- Bénévolat au restaurant populaire de Victoriaville dans le cadre du projet Boîtes à lunch.\\

\end{rSection}

%--------------------------------------------------------------------------------
%    Autres réalisations
%-----------------------------------------------------------------------------------------------

\begin{rSection}{Autres réalisations}

{\bf Session en mobilité internationale} \hfill {\em Août 2024 - Décembre 2024}\\
Universitat Pompeu Fabra, Barcelone\\\\
{\bf Mention d'honneur de la doyenne} \hfill {\em 2023-2024}\\
Université Laval\\\\
{\bf Participation à l'épreuve} {\bf\em Politique active} {\bf aux Jeux} \hfill {\em Septembre 2023 - Janvier 2024}\\
{\bf de la science politique}\\
Délégation de l'Université Laval\\\\
{\bf Attestation de la bourse d'études - mérite d'excellence} \hfill {\em Mai 2022}\\
Cégep de victoriaville\\\\
{\bf Participation à l'activité} {\bf\em Faites vos commissions} {\bf de} {\bf\em Par ici la démocratie}{\bf .} \hfill {\em Mai 2022}\\

\end{rSection}

%----------------------------------------------------------------------------------------
%	Compétences spécifiques
%----------------------------------------------------------------------------------------

\begin{rSection}{Compétences spécifiques}

\begin{tabular}{ @{} >{\bfseries}l @{\hspace{6ex}} l }

Langue \ & Français, Anglais (C2), Espagnol (A2), Catalan (A1)\\
Programmes \ & Bases en LateX et Markdown, R, HTML, Git/Github, Série Microsoft Office \\\\

\end{tabular}

\end{rSection}

%	SECTION des profils
%----------------------------------------------------------------------------------------

\begin{rSection}{Profils}

\begin{tabular}{ @{} >{\bfseries}l @{\hspace{6ex}} l }
LinkedIn \ & https://www.linkedin.com/in/olivier-st-louis-78770a2a7/ \\
Github \ & https://github.com/olivierst-louis \\
\end{tabular}

\end{rSection}
\end{document}